% Created 2025-11-14 Fri 12:41
% Intended LaTeX compiler: pdflatex
\documentclass[a4paper]{article}
\usepackage[utf8]{inputenc}
\usepackage[T1]{fontenc}
\usepackage{graphicx}
\usepackage{longtable}
\usepackage{wrapfig}
\usepackage{rotating}
\usepackage[normalem]{ulem}
\usepackage{amsmath}
\usepackage{amssymb}
\usepackage{capt-of}
\usepackage{hyperref}
\usepackage{times}
\usepackage{amsmath}
\usepackage[utf8]{inputenc}
\usepackage{natbib}
\usepackage{graphicx}
\usepackage{listings}
\lstset{basicstyle=\ttfamily\small}
\author{R}
\date{\today}
\title{Life Insurance Reserves Calculator}
\hypersetup{
 pdfauthor={R},
 pdftitle={Life Insurance Reserves Calculator},
 pdfkeywords={},
 pdfsubject={},
 pdfcreator={},
 pdflang={English}}
\begin{document}

\maketitle
\flushleft
\section{Project Objectives}
\label{sec:org58750d9}

Using mortality tables, we want to project required life insurance reserves. To this end, we will make an industry-standard stochastic interest rates model, simulate a fictional portfolio of policyholders, and use CRAN's lifecontingencies package to make values for \(A_xn\) and \(a\).
\section{The Datasets}
\label{sec:orgfa6e46f}

The first dataset is a mortality table from the Office of National Statistics (ONS) in the UK (Office of National Statistics, 2025). After cleaning it up using dplyr, we obtain mx, qx, lx, dx, and ex values for the UK. Our second dataset are the yield curves from the Bank of England (Bank of England, 2025). For this investigation, we use the UK implied real spot curve since we want to model the real interest rate in the model, and assert that these are the real required amounts to reserve, which would need to match inflation in the future. In our yield curve dataset, we have implied real rates from 2.5 to 40 years.
\section{Stochastic Interest Rate Modelling}
\label{sec:orgdf4671f}

In this investigation, we will be using the Vasicek Model:

\[dr_t = a(b - r)dt + \sigma d W_t\]

Where a is a "speed of reversion", quantifying how fast interest rates return to equilibria, b operates as a long-term mean interest rate, \(\sigma\) acts as the constant of volatility, and \(W_t\) is standard Brownian motion.

\hfill \newline

First we start by putting a lagged, differential, and original 2.5 year real yield curve rate into one dataset, and then regressing the change in the differential interest rate on the current rates. We can rewrite the Vasicek model as:

\[dr_t = ab - ar \cdot dt + \sigma d W_t \]

Compared to our regression model:

\[dr_t = \beta_0 + \beta_1 r \cdot dt + \epsilon \]

Which means that:

\(a\) is just the second coefficient sign swapped,
\(b\) is the first coefficient divided by \(a\),
and \(s\) is the goodness of fit of the residuals.

From our regression, we get that:

\[d r_t = 1.04498(1 - 0.4537031 \text{ } r_t)dt + 0.014356 d W_t  \]

We can now perform the Vasicek simulation

\begin{verbatim}
vasicek_sim <- function(a, b, s, size) {
  dt <- 1
  r <- numeric(size)
  r[1] <- b

  for (i in 2:size) {
    dW_t <- rnorm(1, mean = 0, sd = sqrt(dt))
    r[i] <- r[i-1] + a * (b - r[i-1]) * dt + s * dW_t
  }

  return(r)
}
\end{verbatim}
\captionof{figure}{\label{lst:org0467dee}Vasicek Model R Code}

\hfill \newline
After setting a seed, we can now run a simulation for the next 7 years of interest rates.


\begin{center}
\includegraphics[width=.9\linewidth]{Stochastic_Interest_Rate_Modelling/2025-11-14_09-49-51_screenshot.png}
\caption{Future Interest Rate Simulation}
\end{center}
Armed with a set of interest rate projections and mortality rates, all we need to predict insurance rate reserves for a portfolio of policyholders is a portfolio of policyholders. This can easily be simulated using R's random selection features and Monte Carlo sampling. For our portfolio of policyholders, we simulate:
\begin{enumerate}
\item An ID,
\item their age,
\item their gender,
\item smoking status,
\item the term of their policy,
\item whether they drive,
\item their sum assured (using a log normal wealth simulation),
\item the starting year of their terms (between 2014 and 2024).
\end{enumerate}

We can now calculate insurance premiums for every person on the portfolio using the lifecontingencies model. We know from the actuarial equivalence principle that:

\[P = \frac{S \times A_x }{a_{[x]:n}}\]

Since we have mortality tables, ages, and terms, we can very easily make a premium calculating function that uses lifecontingencies to calculate \(A_x\) and \(a_x\):

\begin{verbatim}
premiumCalculator <- function(benefit, age, term, table) { 

    ## lifecontingencies is happy to calculate A and a for us
    A = Axn(actuarialtable = table, x = age, n = term)

    a = axn(actuarialtable = table, x = age, n = term)

    Premium = benefit * A / a
    return(Premium)   
}
\end{verbatim}
\captionof{figure}{\label{lst:org2cae1de}Premium Calculating function}

\hfill \newline

Since we have both male and female mortality rates, we can also write a small utility to apply specific tables to both genders.

This leaves us with the new portfolio:

\begin{table}[htbp]
\caption{Policyholder Portfolio with Insurance Premia}
\centering
\begin{tabular}{rrllrlrrr}
ID & Age & Gender & Smoker & Policy Term & Driver & Sum Assured & Policy Start Year & Premium\\
\hline
1 & 51 & female & yes & 22 & driver & 131000 & 2024 & 177.0\\
2 & 56 & male & no & 18 & driver & 146000 & 2022 & 295.0\\
3 & 39 & male & no & 11 & driver & 129000 & 2014 & 413.0\\
4 & 43 & female & no & 5 & driver & 139000 & 2015 & 92.6\\
5 & 23 & female & no & 22 & driver & 209000 & 2023 & 26.1\\
\end{tabular}
\end{table}
\section{Calculating General Reserve Requirements}
\label{sec:orgf7e9bac}

We now need to evaluate the value of the reserve, based off the principle that the value of the reserve is the future insurance benefits minus the discounted value of the insurance premia.

\[V_x = S \times A_{x+t} - P_x \times a_{x+t}\]

Where \(V_x\) is the value of the reserve. We can therefore extremely easily write a quick Reserve Calculator in R:

\begin{verbatim}

reserveCalculator <- function(benefit, age, term, t, premium, table) {

  remaining_term <- term - t # remaining policy duration
  new_age <- age + t

  if (remaining_term <= 0) {
    return(0)
  }

  ## Same lifecontingencies utilities as before
  A_remaining <- Axn(table, x = new_age, n = remaining_term) 
  a_remaining <- axn(table, x = new_age, n = remaining_term)

  reserve <- benefit * A_remaining - premium * a_remaining 
  return(reserve)
}
\end{verbatim}
\captionof{figure}{\label{lst:orgbd02212}Reserve Calculator}

\hfill \newline
We can arbitrarily select a policy term of 10 to see the required reserves for the next ten years. Doing this, we find that:

\begin{enumerate}
\item Ten year premium values: £363,600.20
\item Ten year required reserves: £1,077,670
\end{enumerate}

(These may vary depending on how the function is used since the portfolio generation could vary depending on what functions are called after the seed is set.)


\newpage
\section{References}
\label{sec:org1f3e71b}
\noindent
Bank of England (2025). \emph{Latest yield curve data}.

\noindent
Office of National Statistics (2025). \emph{National life tables: UK}.
\end{document}
